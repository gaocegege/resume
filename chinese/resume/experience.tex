%-------------------------------------------------------------------------------
%	SECTION TITLE
%-------------------------------------------------------------------------------
\cvsection{工作经历}


%-------------------------------------------------------------------------------
%	CONTENT
%-------------------------------------------------------------------------------
\begin{cventries}

%---------------------------------------------------------
  \cventry
    {软件工程师} % Job title
    {才云科技} % Organization
    {中国上海} % Location
    {2019 年 4 月至今} % Date(s)
    {
      \begin{cvitems} % Description(s) of tasks/responsibilities
        \item 研究和实现机器学习模型服务
      \end{cvitems}
    }

%---------------------------------------------------------
  \cventry
    {合作研究} % Job title
    {才云科技} % Organization
    {中国上海} % Location
    {2017 年 11 月至 2019 年 3 月} % Date(s)
    {
      \begin{cvitems} % Description(s) of tasks/responsibilities
        \item 研究 Kubernetes 上对于机器学习基准测试的系统 \href{https://github.com/kubeflow/kubebench}{kubeflow/kubebench},研究成果发表在 IEEE AI4I'18 会议,在 KubeCon China 2018 发表时长 30 分钟的演讲。
        \item 实现和维护 Kubeflow 在 Jupyter 上的内核项目 \href{https://github.com/caicloud/ciao}{caicloud/ciao},支持从 Jupyter 中发起分布式机器学习训练任务
        \item 为 \href{https://github.com/kubeflow/kubeflow}{Kubeflow} 社区维护 TensorFlow 分布式训练支持 \href{https://github.com/kubeflow/tf-operator}{kubeflow/tf-operator} 和超参数训练系统 \href{https://github.com/kubeflow/katib}{kubeflow/katib}
        \item 研究分布式机器学习任务在大规模机器集群上的调度,研究成果发表在 ICA3PP'18 会议
      \end{cvitems}
    }

%---------------------------------------------------------
  \cventry
    {Google Summer of Code 学生} % Job title
    {Processing 基金会} % Organization
    {中国上海} % Location
    {2017 年 3 月至 2017 年 9 月} % Date(s)
    {
      \begin{cvitems} % Description(s) of tasks/responsibilities
        \item 参与 Google Summer of Code 活动,本次 GSoC 申请的接收率为 6\%(1318/20651)
        \item 基于 R 语言在 Java 虚拟机上的解释器,设计并实现了 Processing 在 R 语言模式 \href{https://github.com/gaocegege/Processing.R}{Processing.R},获得 94 stars,成为本次编程之夏 star 最多的项目
      \end{cvitems}
    }

  \cventry
    {项目实习生} % Job title
    {摩根士丹利} % Organization
    {中国上海} % Location
    {2017 年 2 月至 2017 年 8 月} % Date(s)
    {
      \begin{cvitems} % Description(s) of tasks/responsibilities
        \item 为开源容器调度管理框架 treadmill 实现与 Kubernetes 类似的调度模型
        \item 调度延迟在 100 节点规模下与原本的调度器相比下降 12\%,但增强了其可配置性,支持对节点上硬件资源的动态监控
      \end{cvitems}
    }

  \cventry
    {合作研究} % Job title
    {才云科技} % Organization
    {中国上海} % Location
    {2015 年 11 月至 2017 年 1 月} % Date(s)
    {
      \begin{cvitems} % Description(s) of tasks/responsibilities
        \item 实现基于 Docker 的持续集成与持续部署系统 \href{https://github.com/caicloud/cyclone}{Cyclone}
      \end{cvitems}
    }

  \cventry
    {实习大数据工程师} % Job title
    {上海触宝信息技术有限公司} % Organization
    {中国上海} % Location
    {2015 年 9 月至 2015 年 9 月} % Date(s)
    {
      \begin{cvitems} % Description(s) of tasks/responsibilities
        \item 移植爬虫代码到新的平台,优化重写部分过期的爬虫
      \end{cvitems}
    }

  \cventry
    {实习 Java 研发工程师} % Job title
    {蚂蚁金服(杭州)网络技术有限公司} % Organization
    {中国浙江杭州} % Location
    {2015 年 7 月至 2015 年 9 月} % Date(s)
    {
      \begin{cvitems} % Description(s) of tasks/responsibilities
        \item 在支付宝国际事业部创新业务组任职,从事海外直购业务
        \item 实现部分包裹清关的逻辑和后台管理的逻辑
      \end{cvitems}
    }

\end{cventries}
