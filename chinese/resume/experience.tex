%-------------------------------------------------------------------------------
%	SECTION TITLE
%-------------------------------------------------------------------------------
\cvsection{工作经历}


%-------------------------------------------------------------------------------
%	CONTENT
%-------------------------------------------------------------------------------
\begin{cventries}

%---------------------------------------------------------
  \cventry
    {机器学习平台组 Research 小组 Technical Lead} % Job title
    {才云科技} % Organization
    {中国上海} % Location
    {2020 年 2 月至今} % Date(s)
    {
      \begin{cvitems} % Description(s) of tasks/responsibilities
        \item 负责公司机器学习平台开源工作组相关事宜。组织周会,协调利益相关方共同推进机器学习平台产品的开源社区版。
        \item 开源基于符合 OCI Distribution Spec 的镜像仓库管理机器学习模型的项目 \href{https://github.com/caicloud/ormb}{caicloud/ormb},支持利用镜像仓库提供的版本化能力分发机器学习模型。同时设计与实现了 \href{https://github.com/caicloud/ormb}{caicloud/ormb} 与开源模型服务项目 KFServing,Seldon Core 的集成,部分工作贡献到 Harbor 上游。在其中参与需求分析,系统设计与评审,单元测试与集成测试,以及版本发布等全部过程。
        \item 负责 Kubernetes 上实现模型转换,模型压缩,模型解析的功能。基于 \href{https://github.com/caicloud/ormb}{caicloud/ormb} 实现对机器学习模型的签名解析,模型格式转换,模型量化压缩等。在其中主要参与需求分析与系统设计工作。
        \item 为了提高集群利用率和任务的鲁棒性,参与实现可容错的分布式训练功能 \href{https://github.com/caicloud/ftlib}{caicloud/ftlib},基于 Gossip 协议进行 membership 的管理,探索对弹性分布式训练的支持,目前这一工作被应用在蚂蚁金服开源项目 ElasticDL 中。在其中主要参与需求分析,系统设计工作。
        \item 为了提高集群硬件资源的利用率,满足客户需求,参与设计 GPU 共享产品特性。支持多个 Pod 共享使用 GPU 资源并且提供一定程度的显存隔离,同时能够与现有监控系统集成,对单个容器进行显存使用情况的监控。在其中主要参与需求分析,系统设计工作。
        \item 为了减少重复的开发工作,提高开发侧能效,调研开源联邦学习 FATE,开源模型服务 KFServing,Seldon Core,开源模型仓库 MLFlow Model Registry 等功能,探索复用开源能力的可行性,指导产品研发与社区版研发的工作。
        \item 为了保证公司机器学习平台产品在分布式模型训练和自动机器学习上的功能,参与 Kubeflow 上游的社区治理工作,担任训练和自动机器学习工作组的 Chair 以及 Technical Lead 社区职位,组织例会,协助制定 Kubeflow 在分布式训练和自动机器学习上的版本规划。维护 \href{https://github.com/kubeflow/tf-operator}{kubeflow/tf-operator} 和 \href{https://github.com/kubeflow/katib}{kubeflow/katib} 等多个社区项目。与社区协作者合作完成论文一篇,保持公司在 Kubeflow 社区的贡献度在全球前五,曾是除谷歌之外全球贡献最多的企业贡献者。
        \item 协调组内其他同事的工作,协助开展其他同事对特征仓库,数据仓库,GPU 亲和性调度等功能的调研工作。
      \end{cvitems}
    }

  \cventry
    {机器学习平台组软件工程师} % Job title
    {才云科技} % Organization
    {中国上海} % Location
    {2019 年 4 月至 2020 年 1 月} % Date(s)
    {
      \begin{cvitems} % Description(s) of tasks/responsibilities
        \item 为了提高产品的差异化优势,基于 Kubeflow 实现了超参数搜索的产品功能。支持并行的超参数搜索,同时单次搜索支持分布式训练,支持对资源的限制与隔离,也支持不同的搜索算法,如贝叶斯优化,随机搜索等。在其中参与需求分析,系统设计与评审,单元测试与集成测试,以及版本发布等全部过程。
        \item 为了提高产品的竞争力,保证产品功能的自包含,参与模型服务产品功能。支持 TensorFlow SavedModel,ONNX 等格式的模型服务能力,同时支持简单的灰度发布,自动扩所容能力。与 GPU 共享功能对接,支持多个模型服务共享 GPU。在其中参与需求分析,系统设计与评审,单元测试与集成测试,以及版本发布等全部过程。
        \item 为了提供更加灵活的分布式训练的能力,重构产品功能,支持多种分布式的训练模式。为后端 API 服务器引入缓存机制,大幅度降低了部分接口的延迟。
        \item 与前线的售前工程师与销售同事配合,为客户提供售前答疑,技术支持服务,协助客户完成专利两篇,帮助定位客户在使用公司机器学习平台产品进行分布式训练时遇到的问题。
      \end{cvitems}
    }

%---------------------------------------------------------
  \cventry
    {机器学习平台组合作研究} % Job title
    {才云科技} % Organization
    {中国上海} % Location
    {2015 年 11 月至 2019 年 3 月} % Date(s)
    {
      \begin{cvitems} % Description(s) of tasks/responsibilities
        \item 为了帮助研究者更好地把机器学习领域的基准测试标准化,设计与实现 Kubernetes 上对于机器学习基准测试的系统 \href{https://github.com/kubeflow/kubebench}{kubeflow/kubebench},研究成果发表在 IEEE AI4I'18 会议,在 KubeCon China 2018 发表时长 30 分钟的演讲。
        \item 为了 Jupyter Notebook 中进行分布式的模型训练,设计与实现了 Kubeflow 在 Jupyter 上的内核项目 \href{https://github.com/caicloud/ciao}{caicloud/ciao}。在其中参与需求分析,系统设计与评审,单元测试与集成测试,以及版本发布等全部过程。
        \item 为 \href{https://github.com/kubeflow/kubeflow}{Kubeflow} 社区维护 TensorFlow 分布式训练支持 \href{https://github.com/kubeflow/tf-operator}{kubeflow/tf-operator} 和超参数训练系统 \href{https://github.com/kubeflow/katib}{kubeflow/katib}。
        \item 研究分布式机器学习任务在大规模机器集群上的调度,研究成果发表在 ICA3PP'18 会议。
        \item 实现基于 Docker 的持续集成与持续部署系统 \href{https://github.com/caicloud/cyclone}{Cyclone}。
      \end{cvitems}
    }

\end{cventries}
