%-------------------------------------------------------------------------------
%	SECTION TITLE
%-------------------------------------------------------------------------------
\cvsection{工作经历}


%-------------------------------------------------------------------------------
%	CONTENT
%-------------------------------------------------------------------------------
\begin{cventries}

  \cventry
    {高级软件工程师} % Job title
    {腾讯云} % Organization
    {中国上海} % Location
    {2020 年 11 月至今} % Date(s)
    {
      \begin{cvitems} % Description(s) of tasks/responsibilities
      \end{cvitems}
    }

%---------------------------------------------------------
  \cventry
    {高级软件工程师} % Job title
    {才云科技(被字节跳动并购)} % Organization
    {中国上海} % Location
    {2019 年 3 月至 2020 年 11 月} % Date(s)
    {
      \begin{cvitems} % Description(s) of tasks/responsibilities
        \item 开源基于符合 OCI Distribution Spec 的镜像仓库管理机器学习模型的项目 \href{https://github.com/caicloud/ormb}{caicloud/ormb},支持利用镜像仓库提供的版本化能力分发机器学习模型。同时设计与实现了 \href{https://github.com/caicloud/ormb}{caicloud/ormb} 与开源模型服务项目 KFServing,Seldon Core 的集成,部分工作贡献到 Harbor 上游。
        \item 为了提高集群利用率和任务的鲁棒性,参与设计可容错的分布式训练功能 \href{https://github.com/caicloud/ftlib}{caicloud/ftlib}的系统设计工作,基于 Gossip 协议进行 membership 的管理,探索对弹性分布式训练的支持,目前这一工作被应用在蚂蚁金服开源项目 ElasticDL 中。
        \item 为了保证公司机器学习平台产品在分布式模型训练和自动机器学习上的功能,参与 Kubeflow 上游的社区治理工作,担任训练和自动机器学习工作组的 Chair 以及 Technical Lead 社区职位,组织例会,协助制定 Kubeflow 在分布式训练和自动机器学习上的版本规划。维护 \href{https://github.com/kubeflow/tf-operator}{kubeflow/tf-operator} 和 \href{https://github.com/kubeflow/katib}{kubeflow/katib} 等多个社区项目。与社区协作者合作完成论文一篇,保持公司在 Kubeflow 社区的贡献度在全球前五,曾是除谷歌之外全球贡献最多的企业贡献者。
        \item 协调组内其他同事的工作,协助开展其他同事对特征仓库,数据仓库,GPU 亲和性调度等功能的调研工作。
      \end{cvitems}
    }

%---------------------------------------------------------
  \cventry
    {合作研究(非全职)} % Job title
    {才云科技(被字节跳动并购)} % Organization
    {中国上海} % Location
    {2015 年 11 月至 2019 年 2 月} % Date(s)
    {
      \begin{cvitems} % Description(s) of tasks/responsibilities
        \item 为了帮助研究者更好地把机器学习领域的基准测试标准化,设计与实现 Kubernetes 上对于机器学习基准测试的系统 \href{https://github.com/kubeflow/kubebench}{kubeflow/kubebench},研究成果发表在 IEEE AI4I'18 会议,在 KubeCon China 2018 发表时长 30 分钟的演讲。
        \item 为了探索在 Jupyter Notebook 中进行分布式的模型训练的可行性,设计与实现了 Kubeflow 在 Jupyter 上的内核项目 \href{https://github.com/caicloud/ciao}{caicloud/ciao}。
        \item 为 \href{https://github.com/kubeflow/kubeflow}{Kubeflow} 社区维护 TensorFlow 分布式训练支持 \href{https://github.com/kubeflow/tf-operator}{kubeflow/tf-operator} 和超参数训练系统 \href{https://github.com/kubeflow/katib}{kubeflow/katib}。
        \item 研究分布式机器学习任务在大规模机器集群上的调度,研究成果发表在 ICA3PP'18 会议。
        \item 实现基于 Docker 的持续集成与持续部署系统 \href{https://github.com/caicloud/cyclone}{Cyclone}。
      \end{cvitems}
    }

\end{cventries}
