%-------------------------------------------------------------------------------
%	SECTION TITLE
%-------------------------------------------------------------------------------
\cvsection{演讲与分享}

\cvsubsection{部分文章}

\begin{cventries}
  \cventry
    {}
    {\href{https://github.com/gaocegege/papers-notebook}{论文阅读笔记}}
    {GitHub}
    {2016 年 11 月至今}
    {
      \begin{cvitems} % Description(s)
        \item 记录了 100 余篇学术论文的阅读笔记,论文的方向多为分布式系统,虚拟化,安全,机器学习,超参数训练,网络模型结构搜索等领域
      \end{cvitems}
    }

  \cventry
    {}
    {\href{http://gaocegege.com/Blog/ormb}{ormb:像管理 Docker 容器镜像一样管理机器学习模型}}
    {\href{http://gaocegege.com}{gaocegege.com}}
    {2020 年 5 月 26 日}
    {
      \begin{cvitems} % Description(s)
        \item {
          这篇文章介绍了如何利用 OCI 兼容的镜像仓库管理机器学习模型
        }
      \end{cvitems}
    }

  \cventry
    {}
    {\href{http://gaocegege.com/Blog/why-do-i-like-ray}{开源史海钩沉系列 [1] Ray:分布式计算框架}}
    {\href{http://gaocegege.com}{gaocegege.com}}
    {2020 年 1 月 20 日}
    {
      \begin{cvitems} % Description(s)
        \item {
          这篇文章介绍了 Ray 的设计与实现
        }
      \end{cvitems}
    }

  \cventry
    {}
    {\href{http://t.cn/Eh7UCQx}{Katib: Kubernetes native 的超参数训练系统}}
    {\href{http://gaocegege.com}{gaocegege.com}}
    {2018 年 3 月 7 日}
    {
      \begin{cvitems} % Description(s)
        \item {
          这篇文章主要介绍了 Katib,一个 Kubernetes Native 的超参数训练系统
        }
      \end{cvitems}
    }
\end{cventries}

\cvsubsection{技术演讲}

%-------------------------------------------------------------------------------
%	CONTENT
%-------------------------------------------------------------------------------
\begin{cventries}

%---------------------------------------------------------
  \cventry
    {华为开发者大会}
    {\href{https://docs.google.com/presentation/d/1GAkh0inBsZHC2XUNde0YCQ7W70r2rA7X5VWsGN1D6KU/edit}{Kubeflow + Volcano 加速机器学习平台容器化进程}}
    {中国上海} % Location
    {2020 年 4 月} % Date(s)
    {
      \begin{cvitems} % Description(s)
        \item {
          本次演讲介绍了 Kubeflow 与 Volcano 的集成能力
        }
      \end{cvitems}
    }

  \cventry
    {人工智能产业发展联盟-开源开放组-第二次会议}
    {\href{https://docs.google.com/presentation/d/1lVmPF7KRtxi9DhBeRi4uU2NALtqNFTS6NhlLeuPDsrs/edit}{The Good, the Bad, and the Ugly of Managing ML Systems with Kubernetes}}
    {中国上海} % Location
    {2019 年 7 月} % Date(s)
    {
      \begin{cvitems} % Description(s)
        \item {
          本次演讲,高策分享了才云科技在利用 Kubernetes 支持机器学习工作负载的方式,以及其中支持比较好的地方,以及会遇到的一些问题
        }
      \end{cvitems}
    }

  \cventry
    {KubeCon China 2018}
    {\href{http://sched.co/FvLV}{对 Kubeflow 上的机器学习工作负载做基准测试}}
    {中国上海} % Location
    {2018 年 11 月} % Date(s)
    {
      \begin{cvitems} % Description(s)
        \item {
          本次演讲,介绍了基于 Kubeflow 的开源基准化工具 Kubebench,其帮助我们通过自动化和一致的规范,更好的理解 Kubernetes 上的 ML 工作量的性能特征。我们还说明我们可以怎样利用来自学术界和工业界的其他基准化成就,如 MLPerf 和 Dawnbench
        }
      \end{cvitems}
    }

  \cventry
    {第十届中国 R 语言会议}
    {\href{http://slides.com/gaocegege/processing-r}{
      Processing.R: 使用 R 语言实现新媒体艺术作品}}
    {中国上海} % Location
    {2017 年 12 月} % Date(s)
    {
      \begin{cvitems} % Description(s)
        \item {
          本次演讲介绍了 Processing.R。通过这一项目,用户可以利用 R 语言进行新媒体艺术作品的创作
        }
      \end{cvitems}
    }

%---------------------------------------------------------
\end{cventries}
