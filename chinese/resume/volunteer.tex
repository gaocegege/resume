%-------------------------------------------------------------------------------
%	SECTION TITLE
%-------------------------------------------------------------------------------
\cvsection{其他经历}


%-------------------------------------------------------------------------------
%	CONTENT
%-------------------------------------------------------------------------------
\begin{cventries}

%---------------------------------------------------------
  \cventry
    {导师} % Job title
    {Google Summer of Code 2020} % Organization
    {中国上海} % Location
    {2020 年 1 月至 2020 年 9 月} % Date(s)
    {
      \begin{cvitems}
        \item 指导学生参与 Kubeflow 社区贡献
      \end{cvitems}
    }

%---------------------------------------------------------
  \cventry
    {导师} % Job title
    {Google Summer of Code 2018} % Organization
    {中国上海} % Location
    {2018 年 1 月至 2018 年 9 月} % Date(s)
    {
      \begin{cvitems}
        \item 指导来自印度的学生 \href{https://github.com/ksdme}{Kilari Teja} 完善与实现 coala Language Server 项目
        \item 在技术选型,代码实现,时间规划等方面进行讨论与指导,定期进行会议交流以及代码 review
        \item 基于 Language Server Protocol 实现的 coala 在 Visual Studio Code 上的插件,共被下载使用 686 次
      \end{cvitems}
    }

  \cventry
    {学生} % Job title
    {Google Summer of Code 2017} % Organization
    {中国上海} % Location
    {2017 年 3 月至 2017 年 9 月} % Date(s)
    {
      \begin{cvitems} % Description(s) of tasks/responsibilities
        \item 参与 Google Summer of Code 活动,本次 GSoC 申请的接收率为 6\%(1318/20651)
        \item 基于 R 语言在 Java 虚拟机上的解释器,设计并实现了 Processing 在 R 语言模式 \href{https://github.com/gaocegege/Processing.R}{Processing.R},获得 114 stars,成为本次编程之夏 star 最多的项目
      \end{cvitems}
    }

  \cventry
    {MOOC \& Open Source 组长} % Job title
    {上海交通大学东岳网络工作室} % Organization
    {中国上海} % Location
    {2016 年 9 月至 2019 年 1 月} % Date(s)
    {
      \begin{cvitems}
        \item 维护东岳网络工作室的技术博客以及知乎专栏:\href{https://zhuanlan.zhihu.com/dongyue}{东岳网络工作室团队}
        \item 维护上海交通大学 XeLaTeX 学位论文模板:\href{htts://github.com/sjtug/sjtuthesis}{SJTUThesis}
        \item 进行定期技术分享,组织成员进行开源社区的贡献活动
      \end{cvitems}
    }

\end{cventries}
